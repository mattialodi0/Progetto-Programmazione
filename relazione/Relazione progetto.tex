\documentclass[12pt]{article}
\usepackage{amsmath}
\usepackage{graphicx}
\usepackage{hyperref}
\usepackage[latin1]{inputenc}

\title{Progetto di Programmazione}
\author{Nome : n. matricola}
\date{settembre 2023}

\begin{document}
\maketitle

\begin{figure}[h]
    \centering
    \includegraphics{raptor.jpg}
    % qui ci va il logo dell'unibo ma ora non me lo fa mettere
\end{figure}

\newpage


\section*{Suddivisione dei compiti}
\subsection*{Davide}
\begin{itemize}
    \item Menu principale
    \item Menu di pausa
\end{itemize}
File: Menu, Menu\_playing
 
\subsection*{Elia}
\begin{itemize}
    \item Personaggio 
    \item Nemici
\end{itemize}
File: Boom, Character, Chaser, Coward, Drunk, Enemy, Flyer, Hero, Projectile, Shooter, Stalker, Time, Turret

\subsection*{Matteo}
\begin{itemize}
    \item Artefatti
    \item Sistema di combattimento
\end{itemize}
File: Artifact, 

\subsection*{Mattia}
\begin{itemize}
    \item Creazione e gestione delle stanze 
    \item Strutture dati per la loro memorizzazione
    \item Grafica
\end{itemize}
File: Board, Templates, GeneralTemplate, Room, Door, Wall

\newpage
\section*{Scelte implementative}
\subsection*{Il Gioco}
La classe Menu gestisce il menù principale, mentre la classe Game si occupa di gestire tutto ciò che riguarda il gioco.
% brutto come è scritto
Game comprende 3 finestre (per la grafica delle stanze, delle staistiche del personaggio e per il punteggio), il personaggio (Hero), l'indice delle stanze e un puntatore alla stanza corrente.

\subsection*{Le Stanze}
Ogni stanza contiene un puntatore alle stanze con cui confina che sono già state esplorate, in modo da potersi muovere in una di queste senza visitare l'indice; delle coordinate x e y per identificarla univocamente all'interno della mappa, dei flag per gestire lo stato di ogni porta (aperta/chiusa presente/non presente), e un template. Il template contiene i muri, le porte, i nemici e gli artefatti.
Le stanze sono memorizzate in una struttura dati di tipo vettore, chiamata indice, in modo da potere essere aggiunte gradualmente esplorando la mappa; l'indice viene usato solo quando si crea una nuova stanza per avere delle porte coerenti con il resto della mappa.
In questo modo il cambio di stanza quando si torna in una stanza già visitata è molto rapido, mentre è necessariamente più lento quando se ne visita una nuova. 

\subsection*{I Tempalte}
Per il contenuto delle stanze si usa una classe che memorizza il contenuto della stanza, detta template, ce ne sono 39, di diverse rarità e ogni volta che si crea una nuova stanza, gliene ne viene asseganto uno.
Per potere essere interscambiabili i template sono tutti sottoclassi di GeneralTemplate e sono memorizzati all'interno della stanza come puntatori.
I muri, le porte, i nemici e gli artefatti sono tutti memorizzati in array dinamici in modo da potere variare in numero e contenuto per ogni stanza, anche tra quelle con lo stesso template.
Il file GeneralTemplate.cpp contiene delle funzioni per aggiungere alle stanze, insiemi di muri o porte in modo che abbiano una certa forma.

\subsection*{Il Personaggio}
\subsection*{I Nemici}
\subsection*{Gli artefatti}
\subsection*{Il sistema di combattimento}

\end{document}












