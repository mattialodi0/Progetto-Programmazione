\documentclass[12pt]{article}
\usepackage{amsmath}
\usepackage{graphicx}
\usepackage{hyperref}
\usepackage[latin1]{inputenc}

\title{Progetto di Programmazione}
\author{Nome : n. matricola}
\date{settembre 2023}

\begin{document}
\maketitle

\begin{figure}[h]
    \centering
    \includegraphics{raptor.jpg}
    % qui ci va il logo dell'unibo ma ora non me lo fa mettere
\end{figure}

\newpage


\section*{Suddivisione dei compiti}
\subsection*{Davide}
\begin{itemize}
    \item Menu principale
    \item Menu di pausa
\end{itemize}
File: Menu, Menu\_playing
 
\subsection*{Elia}
\begin{itemize}
    \item infrastruttura
    \item Nemici 
    \item Sistema di combattimento
    \item Debugging
\end{itemize}
File: main,Game,Boom, Character, Chaser, Coward, Drunk, Enemy, Flyer, Projectile, Shooter, Stalker, Time, Turret

\subsection*{Matteo}
\begin{itemize}
    \item Artefatti
    \item Hero
    \item Sistema di combattimento
\end{itemize}
File: Artifact, Hero

\subsection*{Mattia}
\begin{itemize}
    \item Creazione e gestione delle stanze 
    \item Strutture dati per la loro memorizzazione
    \item Grafica
\end{itemize}
File: Board, Templates, GeneralTemplate, Room, Door, Wall

\newpage
\section*{Scelte implementative}
\subsection*{Il Gioco}
La classe Menu gestisce il menù principale, mentre la classe Game si occupa di gestire tutto ciò che riguarda il gioco.
% brutto come è scritto
Game comprende 3 finestre (per la grafica delle stanze, delle statistiche del personaggio e per il punteggio), il personaggio (Hero), l'indice delle stanze e un puntatore alla stanza corrente.

\subsection*{Le Stanze}
Ogni stanza contiene un puntatore alle stanze con cui confina che sono già state esplorate, in modo da potersi muovere in una di queste senza visitare l'indice; delle coordinate x e y per identificarla univocamente all'interno della mappa, dei flag per gestire lo stato di ogni porta (aperta/chiusa presente/non presente), e un template. Il template contiene i muri, le porte, i nemici e gli artefatti.
Le stanze sono memorizzate in una struttura dati di tipo vettore, chiamata indice, in modo da potere essere aggiunte gradualmente esplorando la mappa; l'indice viene usato solo quando si crea una nuova stanza per avere delle porte coerenti con il resto della mappa.
In questo modo il cambio di stanza quando si torna in una stanza già visitata è molto rapido, mentre è necessariamente più lento quando se ne visita una nuova. 

\subsection*{I Template}
Per il contenuto delle stanze si usa una classe che memorizza il contenuto della stanza, detta template, ce ne sono 39, di diverse rarità e ogni volta che si crea una nuova stanza, gliene ne viene asseganto uno.
Per potere essere interscambiabili i template sono tutti sottoclassi di GeneralTemplate e sono memorizzati all'interno della stanza come puntatori.
I muri, le porte, i nemici e gli artefatti sono tutti memorizzati in array dinamici in modo da potere variare in numero e contenuto per ogni stanza, anche tra quelle con lo stesso template.
Il file GeneralTemplate.cpp contiene delle funzioni per aggiungere alle stanze, insiemi di muri o porte in modo che abbiano una certa forma.

\subsection*{Il Personaggio}
La classe Hero controlla tutte le caratteristiche dell'Eroe: vita, danno, chiave (per aprire porte), reload, range e abilità.
Le caratteristiche dell'eroe prendono valori diversi a seconda della classe che viene scelta prima di iniziare a giocare. Diverse classi possono avere quindi più o meno vita, range, danno e così via.
La funzione "useAbility" attiva abilità differenti a seconda della classe selezionata.
Le abilità possono modificare temporaneamente le caratteristiche dell'eroe o ripristinargli la vita.
La funziona "attack" crea un nuovo proiettile nella direzione voluta. "CenterHero" serve a posizionare il personaggio al centro della stanza quando inizia il gioco. 
Tutte le restanti funzioni modificano le variabili del personaggio. Per esempio "useKey" consuma la chiave del personaggio e apre una porta, "increaseHealth" aumenta la vita.


\subsection*{I Nemici}
I nemici sono Character come l'eroe, ma appartengono alla classe Enemy, che include due funzioni virtual, una per la decisione della direzione da prendere, e una per il controllo dei proiettili che ogni nemico ha.
sono implementati diversi nemici:
Un Drunk che si muove a random e non spara,
Una turret che sta ferma e spara di continuo,
Un Coward che scappa e lancia bombe dietro di se, anche se limitate, coward include anche un sistema avanzato per trovare aperture e scappare senza fermarsi in un angolo, e può lasciare artefatti quando muore,
Un Chaser che ti insegue per colpirti da vicino 
Un Flyer che si comporta come Chaser ma può volare sopra i muri e le porte chiuse
Uno Shooter che ti spara da lontano,
Uno Stalker che diventa invisibile mentre non ti spara,
Un Boom che viene vicino a te ed esplode a kamikaze.
Ogni nemico si attiva solo se è ad una certa distanza ed ha line of sight con il player, ma ha una corta memoria anche se viene persa per periodi brevi, dopo di che torna idle, stando fermo o muovendosi a random.
Dopo aver sparato ogni nemico ha bisogno di un tempo di reload, in cui è vulnerabile, mentre Boom ha un fuse timer prima di esplodere.
Le statistiche di ogni nemico(hp, range dei nemici ranged, e dmg) dipendono dalla difficoltà , che scala direttamente con lo score ottenuto eliminando nemici.
Il proiettile dipende dal nemico che l'ha sparato,con caratteristiche diverse per nemico.
Ogni nemico può essere promosso a Boss, con statistiche maggiorate e che lasciano artefatti quando muoiono.
I nemici possono essere spawnati con tipo scelto o random, con coordinate scelte o random, o qualsiasi combinazione di questi, Drunk e Turret sono gli unici due che a random non possono essere scelti.



\subsection*{Gli artefatti}
Gli Artefatti sono creati come semplici Drawable. Sono quindi dei caratteri presenti nelle stanze di gioco, con cui possiamo interagire. Essi vengono creati insieme al Template della stanza. A differenza dei nemici che vengono creati casualmente, essi vengono posizionati in punti precisi. Questo per non permettere la generazione in luoghi troppo facili da raggiungere.
Esistono quattro tipi di artefatti: Range, Danno, Vita e Chiave.
Ogni artefatto aumenta la statistica associata e cura il giocatore.
Possono rendere il gioco più facile con l'aumentare dello score, ma è bilanciato dall'aumento progressivo della difficoltà.

\subsection*{Il sistema di combattimento}
Il sistema di combattimento regola danno inflitto e danno subito da nemici ed eroe diminuendo gli Hp del Character colpito, che arrivati a 0 lo uccidono. 
L'intero sistema è costruito sul Projectile, che ogni tick che rimane sullo schermo aumenta il proprio Uptime.
Una volta che l'Uptime supera il range associato al proprio nemico o dopo aver colpito qualcosa il proiettile scompare.

\end{document}












